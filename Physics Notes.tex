\documentclass[11pt]{article}

\usepackage{graphicx}
\usepackage{amssymb}
\usepackage{multicol}
\usepackage{amsmath}
\usepackage{savetrees}

\makeatletter
\renewcommand\tableofcontents{%
    \@starttoc{toc}%
}
\makeatother

\newcommand{\Lagr}{\mathcal{L}}					%Lagrangian
\newcommand{\Hami}{\mathcal{H}}					%Hamiltonian

\newcommand{\Imp}{\mathcal{I}}					%Impulse

\newcommand{\degree}{\ensuremath{^\circ}}			%Degree Celcius

\newcommand{\ve}[1]{\ensuremath{\boldsymbol{#1}}}	%Vector
\newcommand{\fve}[1]{\ensuremath{\vec{#1}}}			%Four-vector

\title{Physics Olympiad Notes}
\author{Yuan Chen Yang\\NUS High School}

\begin{document}

\maketitle
\begin{multicols}{2}
\tableofcontents
\end{multicols}
%\pagebreak
\newpage

\begin{multicols*}{3}
%%%%%%%%%%%%%%%%%
\section{Measurement and Uncertainty}
%%%%%%%%%%%%%%%%%
\subsection{Uncertainty in Instruments}
All instruments have uncertainty:
\begin{enumerate}
\item Analogue Instruments: Uncertainty is half the the smallest measurement unit
\item Digital Instruments: Uncertainty is the smallest significant figure
\item Human reaction time: $\pm 0.10$s
\end{enumerate}
%%%%%%%%%%%%%%%%%
\subsection{Significant Figures}
\begin{enumerate}
\item Adding or subtracting: Follow term with least {\em decimal place}
\item Multiplying or Dividing: Follow term with least {\em significant figure}
\end{enumerate}
%%%%%%%%%%%%%%%%%
\subsection{Propagation of error}
For any $f(a, \cdots)$ the general formula for $\Delta f$ is:
	\begin{align*}
	\Delta f = \sqrt{\left( \frac{\partial f}{\partial a} \Delta a \right)^2 + \cdots}
	\end{align*}
Some specific examples:
\begin{enumerate}
\item $f=a\pm b$
	\begin{align*}
	\Delta f = \sqrt{(\Delta a)^2+(\Delta b)^2}
	\end{align*}
\item $f=ab$ or $f=\frac{a}{b}$
	\begin{align*}
	\frac{\Delta f}{f} = \sqrt{\left(\frac{\Delta a}{a} \right)^2+\left(\frac{\Delta b}{b} \right)^2}
	\end{align*}
\end{enumerate}
%%%%%%%%%%%%%%%%%
\section{Mechanics}
%%%%%%%%%%%%%%%%%
\subsection{Statics}
%%%%%%%%%%%%%%%%%
When all objects are motionless (or have constant velocity),
	\begin{align*}
	\sum{\ve{F}_{net}}&=0 \\
	\sum{\ve{\tau}_{net}}&=0
	\end{align*}
Four basic forces to consider:
\begin{description}
\item[Tension] Pulling force felt by a rope, string, etc. Every piece of rope feels a pulling force in both directions.
\item[Friction] Parallel to surface of contact, can be static or kinetic.
\item[Normal] Perpendicular to surface of contact, prevents object from falling through surface.
\item[Gravity] Force acting between two objects with mass. Always acts downwards for objects on surface of earth.
\end{description}
\subsection{Kinematics}
%%%%%%%%%%%%%%%%%
	\begin{align*}
	\ve{v} &= \lim_{\Delta t \rightarrow 0} \frac{\Delta \ve{x}}{\Delta t} = \frac{d\ve{x}}{dt} = 	\ve{\dot{x}} \\
	\ve{a} &= \frac{d\ve{v}}{dt} = \frac{d^2\ve{x}}{dt^2} = \ve{\dot{x}} = \ve{\ddot{x}} 
	\end{align*}
\subsubsection{Polar Coordinates}
%%%%%%%%%%%%%%%%%
Differentiation of unit vectors:
	\begin{align*}
	\dot{\hat{\ve{r}}} &= \dot{\theta} \hat{\ve{\theta}}\\
	\dot{\hat{\ve{\theta}}} &= -\dot{\theta} \hat{\ve{r}}
	\end{align*}
Velocity and acceleration in polar form:
	\begin{align*}
	\ve{r} &= r\hat{\ve{r}} \\
	\ve{v} &= \dot{\ve{r}} = \dot{r} \hat{\ve{r}} + r \dot{\theta} \hat{\ve{\theta}} \\
	\ve{a} &= \dot{\ve{v}} = (\ddot{r} - \dot{\theta}^2 r) \hat{\ve{r}} + (r \ddot{\theta} + 2 \dot{r} \dot{\theta}) \hat{\ve{\theta}}
	\end{align*}
\subsection{Dynamics}
%%%%%%%%%%%%%%%%%
	\begin{align*}
	\ve{F} &= m \ve{\ddot{x}} \\
	\ve{F}_{action} &= - \ve{F}_{reaction}
	\end{align*}
Free body diagram techniques:
\begin{enumerate}
\item $\Sigma \ve{F}_{net} = 0$ for massless pulleys
\item Conservation of string
\end{enumerate}
Solving differential equations in 1-dimension:
\begin{enumerate}
\item $F=f(t)$
	\begin{align*}
	m \int_{v_0}^{v(t)} dv' &= \int_{t_0}^{t} f(t') dt' \\
	m \int_{x_0}^{x(t)} dx' &= \int_{t_0}^{t} v(t') dt' 
	\end{align*}
\item $F=f(x)$
	\begin{align*}
	a= \frac{dv}{dt} = \frac{dv}{dt} \frac{dx}{dx} &= v \frac{dv}{dx} \\
	m \int_{v_0}^{v(x)} v' dv' &= \int_{x_0}^{x} f(x') dx'
	\end{align*}
\item $F=f(v)$
	\begin{align*}
	m \int_{v_0}^{v(t)} \frac{dv'}{f(v')} = \int_{t_0}^{t}dt'
	\end{align*}
\end{enumerate}
\subsubsection{Friction}
%%%%%%%%%%%%%%%%%
Kinetic and static friction:
	\begin{align*}
	\ve{f_k}&=\mu_k\ve{N} \\
	\ve{f_s}&\leq\mu_s\ve{N}
	\end{align*}
Static friction does no work.
\subsubsection{Constraining Forces}
%%%%%%%%%%%%%%%%%
For any rigid body, there are 6 degrees of freedom ($DF$). There can be constraining forces ($C$) acting on the body.
\begin{itemize}
\item Statics: $C+DF=6$
\item Dynamics $C+DF \geq 6$
\end{itemize}
There are 3 assumptions made for a body moving without any constraint:
\begin{enumerate}
\item $\ve{f}_{ij} \parallel \ve{r}_{ij}$
\item $\ve{r}_{ij}$ is constant for any 2 points in a rigid body
\item $\ve{f}_{12} + \ve{f}_{21} = 0$
\end{enumerate}
\subsubsection{Fictitious Forces}
%%%%%%%%%%%%%%%%%
Force felt by an object in a non-inertial frame. Let $\ve{r}$ to be the position vector of the object in the accelerated frame and $\ve{R}$ be the position vector of the accelerated frame, then the possible forces that acts on $\ve{r}$ are:
	\begin{align*}
	\frac{\delta^2\ve{r}}{\delta t ^2} = \frac{\ve{F}}{m}
	&- \frac{d^2\ve{R}}{dt^2}
	- \ve{\omega} \times (\ve{\omega}\times\ve{r})\\
	&- 2\ve{\omega} \times \ve{v}
	- \frac{d\ve{\omega}}{dt} \ve{r}
	\end{align*}
\begin{enumerate}
\item Translational force: $- m\frac{d^2\ve{R}}{dt^2}$
\item Centrifugal force: $-m\ve{\omega} \times (\ve{\omega}\times\ve{r})$
\item Coriolis force: $-2m\ve{\omega} \times \ve{v}$
\item Azimuthal force: $-m \frac{d\ve{\omega}}{dt} \ve{r}$
\end{enumerate}
\subsection{Conservation Laws}
%%%%%%%%%%%%%%%%%
\begin{description}
\item[Energy] $W_{NC} = 0$
\item[Momentum] $\Sigma \ve{F}_{net} = 0$ 
\item[Angular Momentum] $\Sigma \ve{\tau}_{net} = 0$ 
\end{description}
\subsection{Energy}
%%%%%%%%%%%%%%%%%
For a force in one dimension:
	\begin{align*}
	mv\frac{dv}{dx} &= F(x) \\
	\frac{1}{2}mv^2 &= E + \int_{x_0}^{x} F(x') dx'
	\end{align*}
We can then define \emph{potential energy}: 
	\begin{align*}
	U(x) = - \int_{x_0}^x F(x') dx' 
	\end{align*}
Work-Energy theorem:
	\begin{align*}
	W_{AB} &= \int_{x_1}^{x_2} F(x') dx' \\
	W_{\text{total}} &= \Delta KE
	\end{align*}
Conservative forces are forces that only depend on {\em position}. For conservative forces:
	\begin{align*}
	\oint \ve{F} \cdot d\ve{l} &= 0 \\
	\ve{\nabla} \times \ve{F} &= 0 \\
	\ve{F} &= - \ve{\nabla} U \\
	W_{C} &= -\Delta U
	\end{align*}
For non-conservative forces:
	\begin{align*}
	W_{NC} = \Delta(K+U) = \Delta E
	\end{align*}
Where $E$ is defined as the mechanical energy of the system.
\subsubsection{Energy Analysis}
%%%%%%%%%%%%%%%%%
The Lagrangian method is based on the \emph{principle of stationary action}.
	\begin{align*}
	\Lagr(\dot{x},x,t) = T - V \\
	\frac{d}{dt}(\frac{\partial \Lagr}{\partial \dot{x}}) - \frac{\partial \Lagr}{\partial x} = 0
	\end{align*}
The Hamiltonian $\Hami$ can be used for the conservation of energy:
	\begin{align*}
	\Hami(\dot{x},x,t) &= T + V \\
	\dot{\Hami}&=0
	\end{align*}
Where $T$ is the kinetic energy, and $V$ is the potential energy of the system.
\subsubsection{Power}
%%%%%%%%%%%%%%%%%
Power is the rate of work done per unit time:
	\begin{align*}
	P=\frac{dW}{dt}
	\end{align*}
Mechanical power:
	\begin{align*}
	P=\frac{d}{dt}\oint \ve{F} \cdot d\ve{x}&=\frac{d}{dt}\oint \ve{F} \cdot \frac{d\ve{x}}{dt} dt\\
	&=\ve{F} \cdot \ve{v}
	\end{align*}
\subsection{Momentum}
%%%%%%%%%%%%%%%%%
Momentum is defined as:
	\begin{align*}
	\ve{p} = m \ve{v}
	\end{align*}
When there is no net force on the system,
	\begin{align*}
	\sum \ve{F}_{net} = 0 &\Rightarrow \frac{d\ve{p}}{dt} = 0 \\
	&\Rightarrow \ve{p} \text{ is conserved}
	\end{align*}
Impulse is defined as:
	\begin{align*}
	\Imp &= \int_{t_1}^{t_2} \ve{F}(t) dt = \int_{t_1}^{t_2} \frac{d\ve{p}}{dt} dt \\
	\Imp &= \ve{p}(t_2) - \ve{p}(t_1) = \Delta \ve{p}
	\end{align*}
For perfectly elastic collisions of two objects in 1-D, relative velocity is constant. 
	\begin{align*}
	\ve{v}_1 - \ve{v}_2 = - (\ve{v}'_1 - \ve{v}'_2)
	\end{align*}
For other collisions in 1-D, we have the coefficient of restitution $e$:
	\begin{align*}
	e = -\frac{\ve{v}'_2 - \ve{v}'_1}{\ve{v}_2 - \ve{v}_1} \qquad 0 \leq e \leq  1
	\end{align*}
\subsection{Central Forces}
%%%%%%%%%%%%%%%%%
For any particle subjected to a central force,
	\begin{align*}
	F(r) &= m r \dot{\theta}^2 - m \ddot{r} \\
	L &= mr^2\dot{\theta}
	\end{align*}
Because angular momentum $L$ is constant, we can look at central forces systems in 1-dimension.
	\begin{align*}
	V_{\text{eff}}(r) &= \frac{L^2}{2mr^2} + V(r) \\
	E &= V_{\text{eff}} + \frac{1}{2} m \dot{r}^2
	\end{align*}
\subsubsection{Gravity}
%%%%%%%%%%%%%%%%%
For any two point masses of $m_1$ and $m_2$ in empty space, the gravitational force between them is:
	\begin{align*}
	\ve{F}=\frac{Gm_1m_2}{|\ve{r}|^2}\hat{\ve{r}}
	\end{align*}
Where $\ve{r}$ is the position vector of one mass respect to the other, and $G$ is the gravitational constant.
	\begin{align*}
	F=mg
	\end{align*}
\noindent For a mass $m$ at the Earth's surface, where $g=9.81m/s^2$ pointing downwards.
\subsection{Uniform Circular Motion}
For a point mass moving in uniform circular motion, we define:
	\begin{align*}
	\omega=\frac{v}{r}
	\end{align*}
The centripetal acceleration $a$ and the force required to keep the object in its circular path:
	\begin{align*}
	a&=\frac{v^2}{r}=\omega^2r\\
	F&=\frac{mv^2}{r}=m\omega^2r
	\end{align*}
\subsection{Rotational Dynamics (Constant $\ve{\hat{L}}$)}
%%%%%%%%%%%%%%%%%
The angular momentum of a point mass is defined as:
	\begin{align*}
	\ve{L}=\ve{r}\times\ve{p}
	\end{align*}
For a flat object lying on a 2-D plane rotating with angular speed $\omega$:
	\begin{align*}
	\ve{L}=\int\ve{r}\times\ve{p}=\int r^2\omega\ve{\hat{z}}dm
	\end{align*}
If we define the {\em moment of intertia} about the $z$-axis to be $I_z=\int (x^2+y^2)dm$, we have:
	\begin{align*}
	L_z&=I_z\omega\\
	T&=\int\frac{1}{2}m\ve{v}^2=\int\frac{r^2\omega^2}{2}dm\\
	&=\frac{1}{2}I_z\omega^2
	\end{align*}
For the $z$-component of $\ve{L}$ and kinetic energy $T$.
\subsubsection{General Motion}
%%%%%%%%%%%%%%%%%
For an object with a moving center of mass, and rotating at $\omega$ about it, 
	\begin{align*}
	\ve{L}&=\ve{r_\text{CM}}\times\ve{p_\text{CM}}+I_\text{CM}\omega\ve{\hat{z}}\\
	T&=\frac{1}{2}mv_\text{CM}^2+\frac{1}{2}I_\text{CM}\omega^2
	\end{align*}
\subsubsection{Torque}
%%%%%%%%%%%%%%%%%
Torque is defined as:
	\begin{align*}
	\ve{\tau}=\ve{r}\times\ve{F}
	\end{align*}
Using an origin satisfying any of the following conditions to calculate $\ve{L}$,
\begin{enumerate}
\item The origin is the center of mass
\item The origin is not accelerating
\item $(\ve{R}-\ve{r_0})$ is parallel to $\ve{r_0}$, the position of the origin in a fixed coordinate system
\end{enumerate}
	\begin{align*}
	\frac{d\ve{L}}{dt}=\sum \ve{\tau_\text{ext}}
	\end{align*}
When there is no external torque, we have the conservation of angular momentum. 
	\begin{align*}
	\ve{\tau_\text{ext}}=I\alpha
	\end{align*}
Where $\alpha=\frac{d\omega}{dt}$ is the angular acceleration. 
\subsubsection{Angular Impulse}
%%%%%%%%%%%%%%%%%
Angular impulse is defined as:
	\begin{align*}
	\Imp_\theta=\int_{t_1}^{t_2}\ve{\tau}(t)dt=\Delta\ve{L}
	\end{align*}
If we have a force $\ve{F}(t)$ applied at a constant distance $R$ from the origin,
	\begin{align*}
	\ve{\tau}(t)&=\ve{R}\times\ve{F}(t) \\
	\Imp_\theta&=\ve{R}\times\Imp \\
	\Delta\ve{L}&=\ve{R}\times(\Delta\ve{p})
	\end{align*}
\subsubsection{Parallel-axis Theorem}
%%%%%%%%%%%%%%%%%
Let an object of mass $M$ rotate about its center of mass with the same frequency $\omega$ as the center of mass rotates about the origin (with radius $R$):
	\begin{align*}
	L_z=(MR^2+I_\text{CM})\omega
	\end{align*}
Thus if the moment of inertia of an object is $I_0$ about a particular axis, its moment of inertia about a parallel axis separated by $R$ is:
	\begin{align*}
	I=MR^2+I_0
	\end{align*}
\subsubsection{Perpendicular-axis Theorem}
%%%%%%%%%%%%%%%%%
For flat 2-D objects in the $x$-$y$ plane, and orthogonal axes $x$, $y$ and $z$:
	\begin{align*}
	I_z=I_x+I_y
	\end{align*}
\subsubsection{Moments of Inertia}
%%%%%%%%%%%%%%%%%
Center of mass for an object of mass $M$:
	\begin{align*}
	\ve{R_\text{CM}}=\frac{\int\ve{r}dm}{M}
	\end{align*}
	
\section{Special Relativity}
%%%%%%%%%%%%%%%%%
\subsection{Postulates}
\begin{enumerate}
\item The speed of light has the same value in all inertial frames
\item Physical laws remain the same in all inertial frames
\end{enumerate}
\subsection{Kinematics}
%%%%%%%%%%%%%%%%%
\subsubsection{Lorentz Transform}
%%%%%%%%%%%%%%%%%
	\begin{align*}
	x&= \gamma (x' + \beta ct') \\
	y&=y' \\
	z&=z' \\
	ct&= \gamma (\beta x' + ct')
	\end{align*}
Where $\gamma = \frac{1}{\sqrt{1-\frac{v^2}{c^2}}}$ and $\beta = \frac{v}{c}$.
\subsubsection{Fundamental Effects}
%%%%%%%%%%%%%%%%%
\begin{description}
\setlength{\itemsep}{2.5mm}
\item[Length contraction]
	\begin{align*}
	l'=\frac{l}{\gamma}
	\end{align*}
Where $l$ is the proper length.
\item[Time dilation]
	\begin{align*}
	t'=\gamma t
	\end{align*}
Where $t$ is the proper time.
\item [Loss of simultaneity]
	\begin{align*}
	\Delta t = \frac{Lv}{c^2}
	\end{align*}
Two events separated by $L$ and $\Delta t$ in the rest frame will appear simultaneous to an observer moving at $v$.
\item[Longitudinal velocity addition]
	\begin{align*}
	v_x'=\frac{u+v}{1+uv/c^2}
	\end{align*}
Where $u$ is the velocity of an object in the frame traveling at $v$ respect to the lab frame, and $v_x'$ is the $x$-velocity of the object viewed by the lab frame.
\item[Transverse velocity addition]
	\begin{align*}
	v_y'=\frac{u_y}{\gamma_v ( 1+ u_xv/c^2)}
	\end{align*}
Where $u_y$ and $u_x$ are velocity components of an object in the frame traveling at $v$ respect to the lab frame, and $v_y'$ is the $y$-velocity of the object viewed by the lab frame.
\item[Longitudinal Doppler effect]
	\begin{align*}
	f'=f\sqrt{\frac{1+\beta}{1-\beta}}
	\end{align*}
Where $f'$ is the frequency observed of a moving source emitting at frequency $f$ in its rest frame.
\end{description}
\subsubsection{Minkowski Diagrams}
%%%%%%%%%%%%%%%%%
Space-time diagrams with $x$ and $ct$ axes. Some properties are:
\begin{enumerate}
\item Light travels at $45\degree$ to horizontal.
\item $x'$ and $ct'$ axes of another moving frame are $\theta$ to the $x$ and $ct$ axes respectively, with
	\begin{align*}
	\tan(\theta)=\beta
	\end{align*}
\item Units on axes of the moving and stationary frames are related by:
	\begin{align*}
	\frac{x'}{x}=\frac{ct'}{ct}=\sqrt{\frac{1+\beta^2}{1-\beta^2}}
	\end{align*}
\end{enumerate}
\subsection{Dynamics}
%%%%%%%%%%%%%%%%%
\subsubsection{Momentum}
%%%%%%%%%%%%%%%%%
	\begin{align*}
	\ve{p}=\gamma_vm\ve{v}=\frac{m\ve{v}}{\sqrt{1-\frac{v^2}{c^2}}}
	\end{align*}
\subsubsection{Energy}
%%%%%%%%%%%%%%%%%
	\begin{align*}
	E^2=p^2c^2+m^2c^4
	\end{align*}
For massive particles:
	\begin{align*}
	E&=\gamma mc^2=\frac{mc^2}{\sqrt{1-\frac{v^2}{c^2}}}
	\end{align*}
For massless particles(such as photons):
	\begin{align*}
	E=pc=\frac{hc}{\lambda}
	\end{align*}
\subsection{4-vectors}
%%%%%%%%%%%%%%%%%
A 4-vector $\fve{A}=(A_1,A_2,A_3,A_4)$ is a quantity that transforms as follows:
	\begin{align*}
	A_1'&= \gamma(A_1+i\beta A_4)\\
	A_2'&= A_2\\
	A_3'&= A_3\\
	A_4'&= \gamma(A_4-i\beta A_1)
	\end{align*}
The dot product of two 4-vectors is invariant under Lorentz transformations:
	\begin{align*}
	\fve{A}\cdot\fve{B}=\fve{A'}\cdot\fve{B'}
	\end{align*}
\subsubsection{Different 4-vectors}
%%%%%%%%%%%%%%%%%
\begin{description}
\setlength{\itemsep}{-2mm}
\item[4-position] $(dx, dy, dz, icdt)$

4-vectors originate from the invariant interval $ds$.
	\begin{align*}
	\fve{ds}^2&=(dx, dy, dz, icdt)^2 \\
	&=dx^2+dy^2+dz^2-c^2dt^2
	\end{align*}
\item[4-velocity] $\gamma_v(\ve{v},ic)$

To obtain other 4-vectors, we can multiply invariant quantities to the 4-position vector, such as proper time:
	\begin{align*}
	d\tau&=\frac{dt}{\gamma}\\
	\fve{v}&=\frac{ds}{d\tau}\\
	&=\gamma_v\left(\frac{dx}{dt},\frac{dy}{dt},\frac{dz}{dt},ic\right)\\
	&=\gamma_v(\ve{v},ic)
	\end{align*}
\item[4-momentum] $\left(\ve{p},i\frac{E}{c}\right)$

As mass is invariant, 
	\begin{align*}
	\fve{p}&=m\fve{v}\\
	&=(\gamma_vm\ve{v},i\gamma_vmc)\\
	&=\left(\ve{p},i\frac{E}{c}\right)
	\end{align*}
For photons in x-direction, the 4-momentum vector is:
	\begin{align*}
	\fve{p}=\left(\frac{h}{\lambda},0,0,i\frac{h}{\lambda}\right)
	\end{align*}
\item[4-wave] $\left(\ve{k},i\frac{\omega}{c}\right)$

For electromagnetic waves,
	\begin{align*}
	k&=\frac{2\pi}{\lambda}=\frac{\omega}{c}\\
	\ve{p}&=\frac{h}{\lambda}=\hbar\ve{k}\\
	E&=hf=\hbar\omega\\
	\fve{p}&=\hbar\left(\ve{k},i\frac{\omega}{c}\right)\\
	\fve{k}&=\frac{\fve{p}}{\hbar}
	\end{align*}
\item[4-force] $\gamma_v\left(\ve{f},\frac{i}{c}\frac{dE}{dt}\right)$
	\begin{align*}
	\fve{F}&=\frac{d\fve{p}}{d\tau}\\
	&=\gamma_v\left(\ve{f},\frac{d}{dt}\left(i\frac{E}{c}\right)\right)
	\end{align*}
\end{description}

\section{Oscillations and Waves}
%%%%%%%%%%%%%%%%%
Most questions involve solving linear differential equations. For such equations, linear combinations of solutions will also be a solution.
\subsection{Simple Harmonic Motion}
%%%%%%%%%%%%%%%%%
We have a spring force, $F=-kx$.
	\begin{align*}
	\ddot{x}+\omega^2x&=0 \text{, where }\omega=\sqrt{\frac{k}{m}}\\
	x(t)&=A \cos(\omega t+\phi)
	\end{align*}
\subsection{Damped Harmonic Motion}
%%%%%%%%%%%%%%%%%
In addition to the spring force, we now have a drag force $F_f=-bv$, and the total force $F=-kx-b\dot{x}$.
	\begin{align*}
	\ddot{x} + 2\gamma\dot{x}+\omega^2x=0
	\end{align*}
Where $2\gamma=b/m$ and $\omega^2=k/m$. Let $\Omega = \sqrt{\gamma^2-\omega^2}$.
	\begin{align*}
	x(t)=e^{-\gamma t}(Ae^{\Omega t}+Be^{-\Omega t})
	\end{align*}
\begin{description}
\item [Underdamping] $(\Omega^2<0)$
	\begin{align*}
	x(t)&=e^{-\gamma t}(Ae^{i\tilde{\omega}t}+Be^{-i\tilde{\omega}t})\\
	&=e^{-\gamma t}C\cos(\tilde{\omega}t + \phi)
	\end{align*}
Where $\tilde{\omega}=\sqrt{\omega^2-\gamma^2}$. The system will oscillate with its amplitude decreasing over time. The frequency of oscillations will be smaller than in the undamped case.
\item[Overdamping] $(\Omega^2>0)$
	\begin{align*}
	x(t)=Ae^{-(\gamma-\Omega)t}+Be^{-(\gamma+\Omega)t}
	\end{align*}
The system will not oscillate, and the motion will go to zero for large $t$.
\item[Critical damping] $(\Omega^2=0)$

We have $\gamma=\omega$, and:
	\begin{align*}
	\ddot{x}+2\gamma\dot{x}+\gamma^2x=0
	\end{align*}
In this special case, $x=te^{-\gamma t}$ is also a solution:
	\begin{align*}
	x(t)=e^{-\gamma t}(A+Bt)
	\end{align*}
Systems with critical damping go to zero the quickest.
\end{description}
\subsection{Driven Harmonic Motion}
%%%%%%%%%%%%%%%%%
We have to solve differential equations of this form:
	\begin{align*}
	\ddot{x}+2\gamma\dot{x}+ax=\sum_{n=1}^{N}{C_ne^{i\omega_nt}}
	\end{align*}
We first find particular solutions for each $n$, by guessing solutions of the form $x_{p_n}(t)=Ae^{i\omega_nt}$:
	\begin{align*}
	-A{\omega_n}^2+2iA\gamma\omega_n+Aa=C_n \\
	x_{p_n}(t)=\frac{C_n}{-{\omega_n}^2+2i\gamma\omega_n+a}e^{i\omega_nt}
	\end{align*}
Using the superposition principle, the final solution is a linear combination of the general solution and the particular solutions, with the combination constants determined by initial conditions.
\subsection{Coupled Oscillators}
%%%%%%%%%%%%%%%%%
Normal modes are states of a system where all parts are moving with the same frequency. General strategy to find normal modes:
\begin{enumerate}
\item Write down the $n$ equations of motions corresponding to the $n$ degrees of freedom the system has.
\item Substitute $x_i=A_ie^{i\omega t}$ into the differential equations to get a system of linear equations in $A_i$, with $i=1,2,\cdots,n$
\item Non-trivial solutions exist if and only if the determinant of the matrix is zero. Solve for $\omega$, and subsequently find $A_i$
\end{enumerate}
The motion of the system can then be decomposed into linear combinations of its normal modes.
\subsection{Small Oscillations}
%%%%%%%%%%%%%%%%%
For an object at a local minimum of a potential well, we can expand $V(x)$ about the equilibrium point:
	\begin{align*}
	V(x)=&V(x_0)+V'(x_0)(x-x_0)\\
	&+ \frac{1}{2!}V''(x_0)(x-x_0)^2+\cdots
	\end{align*}
As $V(x_0)$ is an additive constant, and $V'(x_0)=0$ by definition of equilibrium, 
	\begin{align*}
	V(x) &\approx \frac{1}{2}V''(x_0)(x-x_0)^2 \\
	F = -\frac{dV}{dx}&=-V''(x_0)(x-x_0) \\
	%m\ddot{x}+V''(x_0)x&=V''(x_0)x_0 \\
	\omega&=\sqrt{\frac{V''(x_0)}{m}}
	\end{align*}
\subsection{Waves}
%%%%%%%%%%%%%%%%%

\section{Optics}
%%%%%%%%%%%%%%%%%
\subsection{Geometric Optics}
%%%%%%%%%%%%%%%%%
Results from Fermat's principle of least time:
	\begin{align*}
	\theta_{incidence}&=\theta_{reflection} \\
	n_1\sin{\theta_1}&=n_2 \sin{\theta_2}
	\end{align*}
Sign convention:
\begin{itemize}
\item Light rays travel from left to right
\item $f$ is positive if surface makes rays more convergent 
\item Distances are measured from the surface (left is negative)
\item $s_o$ is negative for real objects
\item $s_i$ is positive for real images
\item $y$ above optical axis is positive
\end{itemize}
	\begin{align*}
	\frac{1}{f}&=\frac{1}{s_{i}}+\frac{1}{s_{o}} \\
	M&=\frac{y_i}{y_o}=-\frac{s_i}{s_o}
	\end{align*}
For thin lenses and mirrors:
	\begin{align*}
	\frac{1}{f} = \frac{2}{R}
	\end{align*}
For composite thin lenses:
	\begin{align*}
	\frac{1}{f} = (n-1)\left(\frac{1}{R_1} + \frac{1}{R_2}\right)
	\end{align*}
Lens formed by interface of two materials with different $n$:
	\begin{align*}
	\frac{n_2-n_1}{R} = \frac{n_2}{s_i} + \frac{n_1}{s_o}
	\end{align*}
\subsection{Polarization}
%%%%%%%%%%%%%%%%%
For polarized light:
	\begin{align*}
	E&=E_0\cos{\theta} \\
	I&=I_0\cos^2{\theta}
	\end{align*}
For unpolarized light:
	\begin{align*}
	\langle I \rangle=I_0 \langle \cos^2{\theta} \rangle = \frac{I_0}{2}
	\end{align*}
Brewster angle at which all reflected light at an interface is polarized:
	\begin{align*}
	\tan{\theta_i}=\frac{n_t}{n_i}
	\end{align*}
\subsection{Physical Optics}
%%%%%%%%%%%%%%%%%
Interference is the superposition of wave amplitudes when waves overlap.
\subsubsection{Double Slit:}
%%%%%%%%%%%%%%%%%
Occurs when slits are of negligible width, distance between slits comparable to wavelength, such that diffraction effects are insignificant. For bright fringes:
	\begin{align*}
	d\sin{\theta}&=m\lambda \\
	y_m&=R\frac{m\lambda}{d} \qquad m \in \mathbb{Z}
	\end{align*}
For incident medium's refractive index $n_i$, reflection medium's refractive index $n_r$, if $n_i < n_r$, the reflected wave undergoes a $\frac{\pi}{2}$ phase shift.
\subsubsection{Single Slit:}
%%%%%%%%%%%%%%%%%
Occurs when size of slit is comparable to wavelength. Location of dark fringes when wavelets at distance $\frac{a}{2}$ destructively interfere:
	\begin{align*}
	\sin{\theta}&=\frac{m\lambda}{d} \\
	y_m&=x\frac{m\lambda}{a} \qquad m \in \mathbb{Z}
	\end{align*}
\subsubsection{Intensity in Diffraction Patterns}
%%%%%%%%%%%%%%%%%
For double slit interference:
	\begin{align*}
	I = I_{\text{max}} \cos^2\left(\frac{\pi d \sin{\theta}}{\lambda} \right)
	\end{align*}
For single slit diffraction:
	\begin{align*}
	I = I_{\text{max}} \left [\frac{\sin(\pi a \sin \theta / \lambda)}{\pi a \sin \theta / \lambda} \right]^2
	\end{align*}
Double slit including effects of diffraction:
	\begin{align*}
	I = I_{\text{max}} &\cos^2\left(\frac{\pi d \sin{\theta}}{\lambda} \right) \\
	&\cdot  \left [\frac{\sin(\pi a \sin \theta / \lambda)}{\pi a \sin \theta / \lambda} \right]^2
	\end{align*}
\end{multicols*}
\end{document}  