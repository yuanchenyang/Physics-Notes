% Copyright (c) 2011-2013 by Chenyang Yuan. 
% This work is licensed under a Creative Commons Attribution-ShareAlike 3.0 Unported License.
% http://creativecommons.org/licenses/by-sa/3.0/

% For American-friendly paper
\documentclass[8pt, letterpaper]{article} 

\usepackage{amssymb}
\usepackage{siunitx}
\usepackage{amsmath}
\usepackage{bm}
\usepackage{graphicx}

\usepackage{multicol}
\usepackage{savetrees}
\usepackage{hyperref}
\usepackage{etoolbox}

% Compile in CGS units
% To compile in SI units use: togglefalse{cgs}
\newtoggle{cgs}
\toggletrue{cgs}

% Remove table of contents' name
\makeatletter
\renewcommand\tableofcontents{%
  \@starttoc{toc}%
}
\makeatother

% -----------Symbols-----------
\newcommand{\Lagr}{\mathcal{L}}	       % Lagrangian
\newcommand{\Hami}{\mathcal{H}}	       % Hamiltonian
\newcommand{\emf}{\mathcal{E}}	       % EMF
\newcommand{\Imp}{\mathcal{I}}	       % Impulse
\newcommand{\dr}{
  \ensuremath{\text{r}}}               % Displacement r 
\newcommand{\dvr}{
  \ensuremath{\textbf{r}}}             % Displacement vector r 
\newcommand{\dvrhat}{
  \ensuremath{\ve{\hat{\dvr}}}}	       % Displacement unit vector r 

% -----------Functions-----------
\newcommand{\ve}[1]{
  \ensuremath{\bm{#1}}}	               % Vector
\newcommand{\uve}[1]{
  \ensuremath{\bm{\hat{#1}}}}          % Unit Vector
\newcommand{\tensor}[1]{
  \ensuremath{\text{\bf{#1}}}}         % Tensor
\newcommand{\fve}[1]{
  \ensuremath{\vec{#1}}}               % Four-vector
\newcommand{\cc}[1]{
  \ensuremath{#1^{\ast}}}               % Complex Conjugate
\newcommand{\pd}[2]{
  \ensuremath{
    \frac{\partial #1}{\partial #2} }} % Partial Derivative
\newcommand{\operator}[1]{
  \ensuremath{\hat{\text{#1}}}}        % Operator


\title{Physics Notes}
\author{Yuan Chenyang}

\begin{document}
\begin{multicols*}{4}
\section{Thermodynamics}
%%%%%%%%%%%%%%%%%
If two objects are in thermal equilibrium with a third system, then they are in equilibrium with each other. 

\subsection{Thermal Expansion}
%%%%%%%%%%%%%%%%%
For linear expansion, the change in length is:
\begin{align*}
  \Delta L = \alpha L_0 \Delta T
\end{align*}
Where $\alpha$ is the coefficient of linear expansion. For area expansion, use approximately $2 \alpha $. For volume expansion, use approximately $3 \alpha$.

\subsection{Kinetic Theory of Gases}
%%%%%%%%%%%%%%%%%
\subsubsection{Ideal Gas Law}
%%%%%%%%%%%%%%%%%
An ideal gas' molecules are treated as non-interacting point particles. For an ideal gas of $N$ particles at pressure $P$, volume $V$ and temperature $T$:
\begin{align*}
  PV = NK_BT
\end{align*}
For a non-ideal gas, the Van der Waals correction to the ideal gas law is:
\begin{align*}
  \left( P + a \left(\frac{n}{V}\right) ^2 \right) \left( V - bn\right)& \\
  = &nRT
\end{align*}
Where $a$ and $b$ are constants.

\subsubsection{Internal Energy}
%%%%%%%%%%%%%%%%%
Different gases at the same temperature have the same average kinetic energy. Thus we define temperature of a substance to be its average kinetic energy. For a monatomic ideal gas:
\begin{align*}
  \frac{1}{2}m\langle v^2 \rangle = \frac{3}{2}kT
\end{align*}
\noindent
For a gas molecule with $r$ atoms, its total kinetic energy, center of mass kinetic energy and internal vibrational/rotational energy are given by:
\begin{align*}
  E_{\text{Total}} &= \frac{3r}{2}kT \\
  E_{\text{COM}} &= \frac{3}{2}kT \\
  E_{\text{Internal}} &= \frac{3(r-1)}{2}kT 
\end{align*}
\noindent
The equipartition theorem states that each degree of freedom a molecule has contributes an extra $\frac{1}{2}KT$ of kinetic energy.

\subsubsection{Maxwell Distribution}
%%%%%%%%%%%%%%%%%
For an ideal gas, the distrubution of its velocities is:
\begin{align*}
  f(v) = 4 \pi v^2 \left( \frac{m}{2 \pi kT} \right)^{\frac{3}{2}} 
  e^{-\frac{mv^2}{2kT}}
\end{align*}
From this distribution, we can get the average speed of a particle:
\begin{align*}
  \langle v \rangle = \sqrt{\frac{8kT}{\pi m}}
\end{align*}
The most probable velocty is the maximum point of the distribution:
\begin{align*}
  v_{\text{mp}} = \sqrt{\frac{2kT}{m}}
\end{align*}
For any two particles, their average relative speed is:
\begin{align*}
  \langle v_{\text{rel}} \rangle = \sqrt{2} \langle v \rangle = 
  \sqrt{\frac{16kT}{\pi m}}
\end{align*}
From this, we can get the mean free path of a particle, the average distance a particle travels before hitting another particle:
\begin{align*}
  l_m = \frac{1}{4 \pi \sqrt{2} r^2 n}
\end{align*}
Where $n$ is the number density of the particle and $r$ is its radius.

\subsubsection{Diffusion}
%%%%%%%%%%%%%%%%%
For a substance undergoing diffusion due to a concentration gradient $\frac{dc}{dx}$, the diffusive flux $J$ is:
\begin{align*}
  J = D A \frac{dc}{dx}
\end{align*}

\subsection{Heat Teansfer}
%%%%%%%%%%%%%%%%%
For heat transfer through a material with length $l$, area $A$ and thermal conductivity $K$ between two heat reservoirs $T_1 > T_2$:
\begin{align*}
  \frac{dQ}{dt} = \frac{KA (T_1 - T_2)}{l}
\end{align*}
For a blackbody at temperature $T$ radiating heat away:
\begin{align*}
  \frac{dQ}{dt} = \sigma A T^4
\end{align*}
The heat transfered by changing the temperature of a solid of mass $m$ with heat capacity $c$ is:
\begin{align*}
  \Delta Q = mc \Delta T
\end{align*}

\subsection{Thermodynamic Processes}
%%%%%%%%%%%%%%%%%
In all the process described below, the heat $Q$ that goes into the gas is positive, and the work done on the gas $W$ is positive. The first law of thermodynamics states that the change of internal energy $U$ is:
\begin{align*}
  U &= Q + W \\
  U (\gamma - 1) &= NkT
\end{align*}
Where $\gamma = C_p/C_v$ is the ideal gas constant and $C_v = C_p - k$.
\subsubsection{Isochoric}
%%%%%%%%%%%%%%%%%
In this constant volume process:
\begin{align*}
  W &= 0 \\
  Q &= N C_v \Delta T \\
  U &= Q
\end{align*}

\subsubsection{Isobaric}
%%%%%%%%%%%%%%%%%
In a constant pressure volume expansion from $V_1$ to $V_2$:
\begin{align*}
  W &= P(V_1 - V_2) \\
  Q &= N C_p \Delta T \\
  U &= N C_v \Delta T
\end{align*}

\subsubsection{Isothermal}
%%%%%%%%%%%%%%%%%
For an isothermal expansion from $V_1$ to $V_2$:
\begin{align*}
  W &= NkT \ln \left( \frac{V_1}{V_2} \right) \\
  Q &= - W \\
  U &= 0 
\end{align*}

\subsubsection{Adiabatic}
%%%%%%%%%%%%%%%%%
For an adiabatic process,
\begin{align*}
  W &= - \int P dV \\
  Q &= 0 \\
  U &= W
\end{align*}
Integrating the work done, we get the following relation:
\begin{align*}
  PV^{\gamma} = \text{constant}
\end{align*}

\subsection{Heat Engines}
%%%%%%%%%%%%%%%%%
The efficiency of a heat engine that takes in $Q_H$ and gives out $Q_L$ while doing work $W$, its efficiency is given by:
\begin{align*}
  \eta &= \frac{|W|}{|Q_H|} \\
  &= 1 - \frac{|Q_L|}{|Q_H|}
\end{align*}
The efficiency of a heat pump that uses $W$ to pump $Q_L$ from the col reservoir is:
\begin{align*}
  \eta &= \frac{|Q_L|}{|W|}
\end{align*}
All reversible engines operating between the same two temperatures have the same efficiency as a Carnot engine, as you can fit many infinitisimally small Carnot cycles into any reversible cycle:
\begin{align*}
  \eta_{\text{carnot}} = 1 - \frac{T_L}{T_H}
\end{align*}

\subsection{Second Law}
\begin{itemize}
\item A process whose only net result is to take heat from a reservoir and convert it to heat is impossible. 
\item No heat engine can working between two temperatures $T_1$ and $T_2$  can have a higher efficiency than a reversible engine.
\end{itemize}

\subsection{Entropy}
%%%%%%%%%%%%%%%%%
\subsubsection{Macroscopic Def.}
%%%%%%%%%%%%%%%%%
Entropy is the measure of disorder. If heat is added reversibly into a system at temperature $T$, the increase in entropy in the system is:
\begin{align*}
  dS = \frac{dQ}{T}
\end{align*}
Entropy is a state function that doesn't depend on the path travelled. The total entropy change in the system and surroundings for a reversible process is zero. For an irreversible process, the total entropy change is always positive. 
\\
\noindent
At $T=0$, $S=0$. This is the third law of thermodynamics. 
\subsubsection{Microscopic Def.}
%%%%%%%%%%%%%%%%%
Boltzmann defined entropy of a system by counting the number of indistinguishable microstates $w$ inside:
\begin{align*}
  S = k \ln w
\end{align*}
\end{multicols*}
\end{document}
