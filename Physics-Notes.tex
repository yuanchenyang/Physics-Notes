% Copyright (c) 2011-2012 by Chenyang Yuan. 
% This work is made available under the terms of the Creative Commons Attribution-NonCommercial license:
% http://creativecommons.org/licenses/by-nc/3.0/

\documentclass[11pt]{article}

\usepackage{amssymb}
\usepackage{siunitx}
\usepackage{amsmath}
\usepackage{bm}
\usepackage{graphicx}

\usepackage{multicol}
\usepackage{savetrees}
\usepackage{hyperref}


\makeatletter
\renewcommand\tableofcontents{%
    \@starttoc{toc}%
}
\makeatother
% -----------Symbols-----------
\newcommand{\Lagr}{\mathcal{L}}	% Lagrangian
\newcommand{\Hami}{\mathcal{H}}	% Hamiltonian
\newcommand{\emf}{\mathcal{E}}	% EMF
\newcommand{\Imp}{\mathcal{I}}	% Impulse
% -----------Functions-----------
\newcommand{\ve}[1]{
  \ensuremath{\bm{#1}}}	               % Vector
\newcommand{\uve}[1]{
  \ensuremath{\bm{\hat{#1}}}}          % Unit Vector
\newcommand{\tensor}[1]{
  \ensuremath{\text{\bf{#1}}}}         % Tensor
\newcommand{\fve}[1]{
  \ensuremath{\vec{#1}}}               % Four-vector
\newcommand{\dr}{
  \ensuremath{\text{r}}}               % Displacement r 
\newcommand{\dvr}{
  \ensuremath{\textbf{r}}}             % Displacement vector r 
\newcommand{\dvrhat}{
  \ensuremath{\ve{\hat{\dvr}}}}	       % Displacement unit vector r 
\newcommand{\cc}[1]{
  \ensuremath{#1^{\ast}}}               % Complex Conjugate
\newcommand{\pd}[2]{
  \ensuremath{
    \frac{\partial #1}{\partial #2} }} % Partial Derivative
\newcommand{\operator}[1]{
  \ensuremath{\hat{\text{#1}}}}        % Operator


\title{Physics Notes}
\author{Yuan Chenyang}

\begin{document}

\maketitle
\begin{center}
\includegraphics[scale=0.7]{deed.png} 
\end{center}

\begin{multicols*}{2}
\tableofcontents
\end{multicols*}
%\pagebreak
\newpage

\begin{multicols*}{3}
%%%%%%%%%%%%%%%%%
\section{Measurement}
%%%%%%%%%%%%%%%%%
\subsection{Instrument Uncertainty}
All instruments have uncertainties:
\begin{enumerate}
\item Analogue Instruments: Half the smallest measurement unit
\item Digital Instruments: The smallest significant figure
\item Human reaction time: $\pm 0.10$s
\end{enumerate}
%%%%%%%%%%%%%%%%%
\subsection{Significant Figures}
\begin{enumerate}
\item Adding or subtracting: Follow term with least {\em decimal place}
\item Multiplying or Dividing: Follow term with least {\em significant figure}
\end{enumerate}
%%%%%%%%%%%%%%%%%
\subsection{Propagation of error}
For any $f(a, \cdots)$ the general formula for $\Delta f$ is:
\begin{align*}
\Delta f = \sqrt{\left( \pd{f}{a} \Delta a \right)^2 + \cdots}
\end{align*}
Some specific examples:
\begin{enumerate}
\item $f=a\pm b$
\begin{align*}
\Delta f = \sqrt{(\Delta a)^2+(\Delta b)^2}
\end{align*}
\item $f=ab$ or $f=\frac{a}{b}$
\begin{align*}
\frac{\Delta f}{f} = \sqrt{\left(\frac{\Delta a}{a} \right)^2+\left(\frac{\Delta b}{b} \right)^2}
\end{align*}
\end{enumerate}
%%%%%%%%%%%%%%%%%
\section{Mechanics}
%%%%%%%%%%%%%%%%%
\subsection{Statics}
%%%%%%%%%%%%%%%%%
When all objects are motionless (or have constant velocity),
\begin{align*}
\sum{\ve{F}_{net}}&=0 \\
\sum{\ve{\tau}_{net}}&=0
\end{align*}
Four basic forces to consider:
\begin{description}
\item[Tension] Pulling force felt by a rope, string, etc. Every piece of rope feels a pulling force in both directions.
\item[Friction] Parallel to surface of contact, can be static or kinetic.
\item[Normal] Perpendicular to surface of contact, prevents object from falling through surface.
\item[Gravity] Force acting between two objects with mass. Always acts downwards for objects on surface of earth.
\end{description}
\subsection{Kinematics}
%%%%%%%%%%%%%%%%%
\begin{align*}
\ve{v} &= \lim_{\Delta t \rightarrow 0} \frac{\Delta \ve{x}}{\Delta t} = \frac{d\ve{x}}{dt} = 	\ve{\dot{x}} \\
\ve{a} &= \frac{d\ve{v}}{dt} = \frac{d^2\ve{x}}{dt^2} = \ve{\dot{x}} = \ve{\ddot{x}} 
\end{align*}
\subsubsection{Polar Coordinates}
%%%%%%%%%%%%%%%%%
Differentiation of unit vectors:
\begin{align*}
\dot{\uve{r}} &= \dot{\theta} \uve{\theta}\\
\dot{\uve{\theta}} &= -\dot{\theta} \uve{r}
\end{align*}
Velocity and acceleration in polar form:
\begin{align*}
\ve{r} &= r\uve{r} \\
\ve{v} &= \dot{\ve{r}} = \dot{r} \uve{r} + r \dot{\theta} \uve{\theta} \\
\ve{a} &= \dot{\ve{v}} = (\ddot{r} - \dot{\theta}^2 r) \uve{r} + (r \ddot{\theta} + 2 \dot{r} \dot{\theta})\uve{\theta}
\end{align*}
\subsection{Dynamics}
%%%%%%%%%%%%%%%%%
\begin{align*}
\ve{F} &= m \ve{\ddot{x}} \\
\ve{F}_{action} &= - \ve{F}_{reaction}
\end{align*}
Free body diagram techniques:
\begin{enumerate}
\item $\Sigma \ve{F}_{net} = 0$ for massless pulleys
\item Conservation of string
\end{enumerate}
Solving differential equations in 1-dimension:
\begin{enumerate}
\item $F=f(t)$
\begin{align*}
m \int_{v_0}^{v(t)} dv' &= \int_{t_0}^{t} f(t') dt' \\
m \int_{x_0}^{x(t)} dx' &= \int_{t_0}^{t} v(t') dt' 
\end{align*}
\item $F=f(x)$
\begin{align*}
a= \frac{dv}{dt} = \frac{dv}{dt} \frac{dx}{dx} &= v \frac{dv}{dx} \\
m \int_{v_0}^{v(x)} v' dv' &= \int_{x_0}^{x} f(x') dx'
\end{align*}
\item $F=f(v)$
\begin{align*}
m \int_{v_0}^{v(t)} \frac{dv'}{f(v')} = \int_{t_0}^{t}dt'
\end{align*}
\end{enumerate}
\subsubsection{Friction}
%%%%%%%%%%%%%%%%%
Kinetic and static friction:
\begin{align*}
\ve{f_k}&=\mu_k\ve{N} \\
\ve{f_s}&\leq\mu_s\ve{N}
\end{align*}
Static friction does no work.
\subsubsection{Constraining Forces}
%%%%%%%%%%%%%%%%%
For any rigid body, there are 6 degrees of freedom ($DF$). There can be constraining forces ($C$) acting on the body.
\begin{itemize}
\item Statics: $C+DF=6$
\item Dynamics $C+DF \geq 6$
\end{itemize}
There are 3 assumptions made for a body moving without any constraint:
\begin{enumerate}
\item $\ve{f}_{ij} \parallel \ve{r}_{ij}$
\item $\ve{r}_{ij}$ is constant for any 2 points in a rigid body
\item $\ve{f}_{12} + \ve{f}_{21} = 0$
\end{enumerate}
\subsubsection{Fictitious Forces}
%%%%%%%%%%%%%%%%%
For any vector $\ve{A}$ in a moving frame, we calculate its time derivative in a frame rotating at $\omega$ respect to the stationary frame:
\begin{align*}
\frac{d\ve{A}}{dt}_{\text{stat}} = \frac{d\ve{A}}{dt}_{\text{mov}} + \ve{\omega} \times \ve{A}
\end{align*}
\noindent
Let $\ve{r}$ be the position vector of the object in an accelerated frame and $\ve{R}$ be the vector to the origin of the accelerated frame, then the possible forces that acts on $\ve{r}$ in the moving frame are:
\begin{align*}
\frac{d^2\ve{r}}{dt ^2} = \frac{\ve{F}}{m}
&- \frac{d^2\ve{R}}{dt^2}
- \ve{\omega} \times (\ve{\omega}\times\ve{r})\\
&- 2\ve{\omega} \times \ve{v}
- \frac{d\ve{\omega}}{dt} \ve{r}
\end{align*}
\begin{enumerate}
\item Translational force: $- m\frac{d^2\ve{R}}{dt^2}$
\item Centrifugal force: $-m\ve{\omega} \times (\ve{\omega}\times\ve{r})$
\item Coriolis force: $-2m\ve{\omega} \times \ve{v}$
\item Azimuthal force: $-m \frac{d\ve{\omega}}{dt} \ve{r}$
\end{enumerate}
\subsection{Conservation Laws}
%%%%%%%%%%%%%%%%%
\begin{description}
\item[Energy] $W_{NC} = 0$
\item[Momentum] $\Sigma \ve{F}_{net} = 0$ 
\item[Angular Momentum] $\Sigma \ve{\tau}_{net} = 0$ 
\end{description}
\subsection{Energy}
%%%%%%%%%%%%%%%%%
For a force in one dimension:
\begin{align*}
m \dot{\ve{r}} \frac{d\dot{\ve{r}}}{d\ve{r}} &= \ve{F}(\ve{r}) \\
\frac{1}{2}m |\dot{\ve{r}}|^2 &= E + \int_{\ve{r_0}}^{\ve{r}} \ve{F}(\ve{r}') \cdot d\ve{r}'
\end{align*}
We can then define \emph{potential energy}: 
\begin{align*}
U(\ve{r}) = - \int_{\ve{r_0}}^{\ve{r}} F(\ve{r}') \cdot d\ve{r}' 
\end{align*}
Work-Energy theorem:
\begin{align*}
W_{AB} &= \int_{\ve{r_1}}^{\ve{r_2}} F(\ve{r}') \cdot d\ve{r}' \\
W_{\text{total}} &= \Delta KE
\end{align*}
Conservative forces are forces that only depend on {\em position}. For conservative forces:
\begin{align*}
\oint \ve{F} \cdot d\ve{r} &= 0 \\
\ve{\nabla} \times \ve{F} &= 0 \\
\ve{F} &= - \ve{\nabla} U \\
W_{C} &= -\Delta U
\end{align*}
For non-conservative forces:
\begin{align*}
W_{NC} = \Delta(K+U) = \Delta E
\end{align*}
Where $E$ is defined as the mechanical energy of the system.
\subsubsection{Energy Analysis}
%%%%%%%%%%%%%%%%%
The Lagrangian method is based on the \emph{principle of stationary action}.
\begin{align*}
\Lagr(\dot{x},x,t) = T - V \\
\frac{d}{dt}\left( \pd{\Lagr}{\dot{x}} \right) - \pd{\Lagr}{x} = 0
\end{align*}
The Hamiltonian $\Hami$ can be used for the conservation of energy:
\begin{align*}
\Hami(\dot{x},x,t) &= T + V \\
\dot{\Hami}&=0
\end{align*}
Where $T$ is the kinetic energy, and $V$ is the potential energy of the system.
\subsubsection{Power}
%%%%%%%%%%%%%%%%%
Power is the rate of work done per unit time:
\begin{align*}
P=\frac{dW}{dt}
\end{align*}
Mechanical power:
\begin{align*}
P=\frac{d}{dt}\oint \ve{F} \cdot d\ve{r}&=\frac{d}{dt}\oint \ve{F} \cdot \frac{d\ve{r}}{dt} dt\\
&=\ve{F} \cdot \ve{v}
\end{align*}
\subsection{Momentum}
%%%%%%%%%%%%%%%%%
Momentum is defined as:
\begin{align*}
\ve{p} = m \ve{v}
\end{align*}
When there is no net force on the system,
\begin{align*}
\sum \ve{F}_{net} = 0 &\Rightarrow \frac{d\ve{p}}{dt} = 0 \\
&\Rightarrow \ve{p} \text{ is conserved}
\end{align*}
Impulse is defined as:
\begin{align*}
\Imp &= \int_{t_1}^{t_2} \ve{F}(t) dt = \int_{t_1}^{t_2} \frac{d\ve{p}}{dt} dt \\
\Imp &= \ve{p}(t_2) - \ve{p}(t_1) = \Delta \ve{p}
\end{align*}
For perfectly elastic collisions of two objects in 1-D, relative velocity is constant. 
\begin{align*}
\ve{v}_1 - \ve{v}_2 = - (\ve{v}'_1 - \ve{v}'_2)
\end{align*}
For other collisions in 1-D, we have the coefficient of restitution $e$:
\begin{align*}
e = -\frac{\ve{v}'_2 - \ve{v}'_1}{\ve{v}_2 - \ve{v}_1} \qquad 0 \leq e \leq  1
\end{align*}
\subsection{Central Forces}
%%%%%%%%%%%%%%%%%
For any particle subjected to a central force,
\begin{align*}
F(r) &= m \ddot{r} - m r \dot{\theta}^2\\
L &= mr^2\dot{\theta}
\end{align*}
Because angular momentum $L$ is constant, we can look at central forces systems in 1-dimension.
\begin{align*}
V_{\text{eff}}(r) &= \frac{L^2}{2mr^2} + V(r) \\
E &= V_{\text{eff}} + \frac{1}{2} m \dot{r}^2
\end{align*}
\subsubsection{Gravity}
%%%%%%%%%%%%%%%%%
For any two point masses of $m_1$ and $m_2$ in empty space, the gravitational force between them is:
\begin{align*}
\ve{F}=\frac{Gm_1m_2}{|\ve{r}|^2}\uve{r}
\end{align*}
Where $\ve{r}$ is the position vector of one mass respect to the other, and $G$ is the gravitational constant.
\begin{align*}
F=mg
\end{align*}
\noindent For a mass $m$ at the Earth's surface, where $g=9.81m/s^2$ pointing downwards.
\subsection{Uniform Circular Motion}
For a point mass moving in uniform circular motion, we define:
\begin{align*}
\omega=\frac{v}{r}
\end{align*}
The centripetal acceleration $a$ and the force required to keep the object in its circular path:
\begin{align*}
a&=\frac{v^2}{r}=\omega^2r\\
F&=\frac{mv^2}{r}=m\omega^2r
\end{align*}
\subsection{\texorpdfstring{Rotational Dynamics (Constant $\uve{L}$)}{Rotational Dynamics (Constant Direction of L)}} %Prevents \hyperref error
%%%%%%%%%%%%%%%%%
\subsubsection{Angular Momentum}
%%%%%%%%%%%%%%%%%
The angular momentum of a point mass is defined as:
\begin{align*}
\ve{L}=\ve{r}\times\ve{p}
\end{align*}
For a flat object lying on a 2-D plane rotating with angular speed $\omega$:
\begin{align*}
\ve{L}=\int\ve{r}\times\ve{p}=\int r^2\omega \uve{z}dm
\end{align*}
If we define the {\em moment of intertia} about the $z$-axis to be $I_z=\int (x^2+y^2)dm$, we have:
\begin{align*}
L_z&=I_z\omega\\
T&=\int\frac{1}{2}m\ve{v}^2=\int\frac{r^2\omega^2}{2}dm\\
&=\frac{1}{2}I_z\omega^2
\end{align*}
For the $z$-component of $\ve{L}$ and kinetic energy $T$.
\subsubsection{General Motion}
%%%%%%%%%%%%%%%%%
For an object with a moving center of mass, and rotating at $\omega$ about it, 
\begin{align*}
\ve{L}&=\ve{r_\text{CM}}\times\ve{p_\text{CM}}+I_\text{CM}\omega \uve{z}\\
T&=\frac{1}{2}mv_\text{CM}^2+\frac{1}{2}I_\text{CM}\omega^2
\end{align*}
\subsubsection{Torque}
%%%%%%%%%%%%%%%%%
Torque is defined as:
\begin{align*}
\ve{\tau}=\ve{r}\times\ve{F}
\end{align*}
Using an origin satisfying any of the following conditions to calculate $\ve{L}$,
\begin{enumerate}
\item The origin is the center of mass
\item The origin is not accelerating
\item $(\ve{R}-\ve{r_0})$ is parallel to $\ve{r_0}$, the position of the origin in a fixed coordinate system
\end{enumerate}
\begin{align*}
\frac{d\ve{L}}{dt}=\sum \ve{\tau_\text{ext}}
\end{align*}
When there is no external torque, we have the conservation of angular momentum. 
\begin{align*}
\ve{\tau_\text{ext}}=I\alpha
\end{align*}
Where $\alpha=\frac{d\omega}{dt}$ is the angular acceleration. 
\subsubsection{Angular Impulse}
%%%%%%%%%%%%%%%%%
Angular impulse is defined as:
\begin{align*}
\Imp_\theta=\int_{t_1}^{t_2}\ve{\tau}(t)dt=\Delta\ve{L}
\end{align*}
If we have a force $\ve{F}(t)$ applied at a constant distance $R$ from the origin,
\begin{align*}
\ve{\tau}(t)&=\ve{R}\times\ve{F}(t) \\
\Imp_\theta&=\ve{R}\times\Imp \\
\Delta\ve{L}&=\ve{R}\times(\Delta\ve{p})
\end{align*}
\subsubsection{Parallel-axis Theorem}
%%%%%%%%%%%%%%%%%
Let an object of mass $M$ rotate about its center of mass with the same frequency $\omega$ as the center of mass rotates about the origin (with radius $R$):
\begin{align*}
L_z=(MR^2+I_\text{CM})\omega
\end{align*}
Thus if the moment of inertia of an object is $I_0$ about a particular axis, its moment of inertia about a parallel axis separated by $R$ is:
\begin{align*}
I=MR^2+I_0
\end{align*}
\subsubsection{Perpendicular-axis Theorem}
%%%%%%%%%%%%%%%%%
For flat 2-D objects in the $x$-$y$ plane, and orthogonal axes $x$, $y$ and $z$:

\begin{align*}
I_z=I_x+I_y
\end{align*}
\subsubsection{Moments of Inertia}
%%%%%%%%%%%%%%%%%
Center of mass for an object of mass $M$:
\begin{align*}
\ve{R_\text{CM}}=\frac{\int\ve{r}dm}{M}
\end{align*}
Common moments of inertia (taken about center of mass unless stated):
\begin{enumerate}
\setlength{\itemsep}{2mm}
\item Point mass at $r$ from axis: $mr^2$
\item Rod of length $L$ about center: $\frac{1}{13}mL^2$
\item Rod of length $L$ about one end: $\frac{1}{3}mL^2$
\item Solid disk of radius $r$ perpendicular to axis: $\frac{1}{2}mr^2$
\item Hollow sphere with radius $r$: $\frac{2}{3}mr^2$
\item Solid sphere with radius $r$: $\frac{2}{5}mr^2$
\end{enumerate}

\subsection{General Rotational Motion}
%%%%%%%%%%%%%%%%%
For any body moving in space, its motion can be written as a sum of its translational motion and a rotation about an axis at a particular time. 

\subsubsection{Angular Velocity}
%%%%%%%%%%%%%%%%%
The angular velocity vector $\ve{\omega}$ points along the axis of rotation, with a  magnitude equal to its angular speed. Its direction is determined by convention of the right hand rule. For an object rotating at $\ve{\omega}$, the velocity of a point at $\ve{r}$ is:
\begin{align*}
\ve{v} = \ve{\omega} \times \ve{r}
\end{align*}
Angular velocities add like vectors. Let $S_1$, $S_2$ and $S_3$ be coordinate systems. If $S_1$ rotates with $\ve{\omega}_{1, 2}$ with respect to $S_2$, and $S_2$ rotates with $\ve{\omega}_{2, 3}$ with respect to $S_3$, then $S_1$ rotates instantaneously with respect to $S_3$ at: 
\begin{align*}
\ve{\omega}_{1, 3} = \ve{\omega}_{1, 2} + \ve{\omega}_{2, 3}
\end{align*}

\subsubsection{Angular Momentum}
%%%%%%%%%%%%%%%%%
\begin{align*}
\ve{L} &= \int \ve{r} \times (\ve{\omega} \times \ve{r}) dm \\
&= \tensor{I} \ve{\omega}
\end{align*}
$\tensor{I}$ is the moment of inertia tensor:
\vspace*{-0.7em}
\begin{center}
\resizebox{\hsize}{!}{%
$\begin{pmatrix}
\int(y^2 +z^2)	& -\int xy        & -\int zx \\
- \int xy       & \int(z^2 + x^2) & -\int yz \\
- \int zx	& -\int yz        & \int(x^2 + y^2) \\
\end{pmatrix}$%
}
\end{center}
The kinetic energy of the object is given by:
\begin{align*}
T &= \int\frac{1}{2} ||\ve{\omega} \times \ve{r}||^2 dm \\
&= \frac{1}{2} \ve{\omega} \cdot \tensor{I}\ve{\omega} = \frac{1}{2} \ve{\omega} \cdot \ve{L}
\end{align*}
To find the angular momentum for an object of mass $M$ in general motion, let the position of its center of mass be $\ve{R}$, its velocity be $\ve{V}$. Then:
\begin{align*}
\ve{L} = M(\ve{R} \times \ve{V}) + \ve{L}_{\text{CM}}
\end{align*}
The kinetic energy of the object is:
\begin{align*}
T = \frac{1}{2}MV^2 + \frac{1}{2}\ve{\omega}' \ve{L}_{\text{CM}}
\end{align*}
Where $\ve{\omega}'$ and $\ve{L}_{\text{CM}}$ are measured about the center of mass along axes parallel to the fixed-frame axes.

\subsubsection{Principal Axes}
%%%%%%%%%%%%%%%%%
A principal axis is an axis of rotation $\uve{\omega}$ such that $\tensor{I}\uve{\omega} = I\uve{\omega}$. An object can rotate about a principal axis at constant angular velocity with no external torque. An orthonormal set of principle axis exists for every object.

\section{Special Relativity}
%%%%%%%%%%%%%%%%%
\subsection{Postulates}
\begin{enumerate}
\item The speed of light has the same value in all inertial frames
\item Physical laws remain the same in all inertial frames
\end{enumerate}
\subsection{Kinematics}
%%%%%%%%%%%%%%%%%
\subsubsection{Lorentz Transform}
%%%%%%%%%%%%%%%%%
\begin{align*}
x&= \gamma (x' + \beta ct') \\
y&=y' \\
z&=z' \\
ct&= \gamma (\beta x' + ct')
\end{align*}
Where $\gamma = \frac{1}{\sqrt{1-\frac{v^2}{c^2}}}$ and $\beta = \frac{v}{c}$.
\subsubsection{Fundamental Effects}
%%%%%%%%%%%%%%%%%
\begin{description}
\setlength{\itemsep}{2.5mm}
\item[Length contraction]
\begin{align*}
l'=\frac{l}{\gamma}
\end{align*}
Where $l$ is the proper length.
\item[Time dilation]
\begin{align*}
t'=\gamma t
\end{align*}
Where $t$ is the proper time.
\item [Loss of simultaneity]
\begin{align*}
\Delta t = \frac{Lv}{c^2}
\end{align*}
Two events separated by $L$ and $\Delta t$ in the rest frame will appear simultaneous to an observer moving at $v$.
\item[Longitudinal velocity addition]
\begin{align*}
v_x'=\frac{u+v}{1+uv/c^2}
\end{align*}
Where $u$ is the velocity of an object in the frame traveling at $v$ respect to the lab frame, and $v_x'$ is the $x$-velocity of the object viewed by the lab frame.
\item[Transverse velocity addition]
\begin{align*}
v_y'=\frac{u_y}{\gamma_v ( 1+ u_xv/c^2)}
\end{align*}
Where $u_y$ and $u_x$ are velocity components of an object in the frame traveling at $v$ respect to the lab frame, and $v_y'$ is the $y$-velocity of the object viewed by the lab frame.
\item[Longitudinal Doppler effect]
\begin{align*}
f'=f\sqrt{\frac{1+\beta}{1-\beta}}
\end{align*}
Where $f'$ is the frequency observed of a moving source emitting at frequency $f$ in its rest frame.
\end{description}
\subsubsection{Minkowski Diagrams}
%%%%%%%%%%%%%%%%%
Space-time diagrams with $x$ and $ct$ axes. Some properties are:
\begin{enumerate}
\item Light travels at $\ang{45}$ to horizontal.
\item $x'$ and $ct'$ axes of another moving frame are $\theta$ to the $x$ and $ct$ axes respectively, with
\begin{align*}
\tan(\theta)=\beta
\end{align*}
\item Units on axes of the moving and stationary frames are related by:
\begin{align*}
\frac{x'}{x}=\frac{ct'}{ct}=\sqrt{\frac{1+\beta^2}{1-\beta^2}}
\end{align*}
\end{enumerate}
\subsection{Dynamics}
%%%%%%%%%%%%%%%%%
\subsubsection{Momentum}
%%%%%%%%%%%%%%%%%
\begin{align*}
\ve{p}=\gamma_vm\ve{v}=\frac{m\ve{v}}{\sqrt{1-\frac{v^2}{c^2}}}
\end{align*}
\subsubsection{Energy}
%%%%%%%%%%%%%%%%%
\begin{align*}
E^2=p^2c^2+m^2c^4
\end{align*}
For massive particles:
\begin{align*}
E&=\gamma mc^2=\frac{mc^2}{\sqrt{1-\frac{v^2}{c^2}}}
\end{align*}
For massless particles(such as photons):
\begin{align*}
E=pc=\frac{hc}{\lambda}
\end{align*}
\subsection{4-vectors}
%%%%%%%%%%%%%%%%%
A 4-vector $\fve{A}=(A_1,A_2,A_3,A_4)$ is a quantity that transforms as follows:
\begin{align*}
A_1'&= \gamma(A_1+i\beta A_4)\\
A_2'&= A_2\\
A_3'&= A_3\\
A_4'&= \gamma(A_4-i\beta A_1)
\end{align*}
The dot product of two 4-vectors is invariant under Lorentz transformations:
\begin{align*}
\fve{A}\cdot\fve{B}=\fve{A'}\cdot\fve{B'}
\end{align*}
\subsubsection{Different 4-vectors}
%%%%%%%%%%%%%%%%%
\begin{description}
\setlength{\itemsep}{-2mm}
\item[4-position] $(dx, dy, dz, icdt)$

4-vectors originate from the invariant interval $ds$.
\begin{align*}
\fve{ds}^2&=(dx, dy, dz, icdt)^2 \\
&=dx^2+dy^2+dz^2-c^2dt^2
\end{align*}
\item[4-velocity] $\gamma_v(\ve{v},ic)$

To obtain other 4-vectors, we can multiply invariant quantities to the 4-position vector, such as proper time:
\begin{align*}
d\tau&=\frac{dt}{\gamma}\\
\fve{v}&=\frac{ds}{d\tau}\\
&=\gamma_v\left(\frac{dx}{dt},\frac{dy}{dt},\frac{dz}{dt},ic\right)\\
&=\gamma_v(\ve{v},ic)
\end{align*}
\item[4-momentum] $\left(\ve{p},i\frac{E}{c}\right)$

As mass is invariant, 
\begin{align*}
\fve{p}&=m\fve{v}\\
&=(\gamma_vm\ve{v},i\gamma_vmc)\\
&=\left(\ve{p},i\frac{E}{c}\right)
\end{align*}
For photons in x-direction, the 4-momentum vector is:
\begin{align*}
\fve{p}=\left(\frac{h}{\lambda},0,0,i\frac{h}{\lambda}\right)
\end{align*}
\item[4-wave] $\left(\ve{k},i\frac{\omega}{c}\right)$

For electromagnetic waves,
\begin{align*}
k&=\frac{2\pi}{\lambda}=\frac{\omega}{c}\\
\ve{p}&=\frac{h}{\lambda}=\hbar\ve{k}\\
E&=hf=\hbar\omega\\
\fve{p}&=\hbar\left(\ve{k},i\frac{\omega}{c}\right)\\
\fve{k}&=\frac{\fve{p}}{\hbar}
\end{align*}
\item[4-force] $\gamma_v\left(\ve{f},\frac{i}{c}\frac{dE}{dt}\right)$
\begin{align*}
\fve{F}&=\frac{d\fve{p}}{d\tau}\\
&=\gamma_v\left(\ve{f},\frac{d}{dt}\left(i\frac{E}{c}\right)\right)
\end{align*}
\end{description}

\section{Electricity and Magnetism}
%%%%%%%%%%%%%%%%%
\subsection{Electrostatics}
%%%%%%%%%%%%%%%%%
\begin{description}
\item[Coulomb's law] The force between a point charge $q$ and test charge $Q$:
\begin{align*}
\ve{F} =\frac{1}{4\pi\epsilon_0}\frac{Qq}{\dr^2}\dvrhat
\end{align*}
Where $\dvr=\ve{r}-\ve{r'}$ is the displacement vector from $Q$ at $\ve{r}$ and $q$ at $\ve{r'}$.
\item[Superposition principle] The interaction between any two charges is unaffected by any other charges
\end{description}

\subsubsection{Electric Field}
%%%%%%%%%%%%%%%%%
The electric field of a point charge is defined as:
\begin{align*}
\ve{E}=\frac{\ve{F}}{Q}=\frac{1}{4\pi\epsilon_0}\frac{q}{\dr^2}\dvrhat
\end{align*}
For a continuous volume charge distribution $\rho(\ve{r'})$, we can use the superposition principle to get:
\begin{align*}
\ve{E}(\ve{r})=\frac{1}{4\pi\epsilon_0}\int_\mathcal{V}\frac{\rho(\ve{r'})}{\dr^2}\dvrhat d\tau'
\end{align*}
Taking the divergence of $\ve{E}$, we get Gauss' law:
\begin{align*}
\ve{\nabla}\cdot\ve{E}&=\frac{\rho{(\ve{r})}}{\epsilon_0}\\
\oint_\mathcal{S}\ve{E}\cdot d\ve{a}&=\frac{Q_\text{enc}}{\epsilon_0}
\end{align*}
Taking the curl of $\ve{E}$:
\begin{align*}
\ve{\nabla}\times\ve{E}&=0\\
\oint\ve{E}\cdot d\ve{l}&=0
\end{align*}
For any surface charge in an electric field $\ve{E}$, the field felt by an area element on the surface is:
\begin{align*}
\ve{E}_\text{felt}=\frac{1}{2}\left(\ve{E}_\text{above}+\ve{E}_\text{below}\right)
\end{align*}
\subsubsection{Electric Potential}
%%%%%%%%%%%%%%%%%
As the line integral of the electrostatic field is path independent, we can define the potential at a point $\ve{r}$:
\begin{align*}
V(\ve{r})=-\int_{\mathcal{O}}^{\ve{r}}\ve{E}\cdot d\ve{l}
\end{align*}
Where $\mathcal{O}$ is a standard reference point, usually set to infinity. The potential of a point charge can then be found, and with the superposition principle we can find the potential of any charge distribution:
\begin{align*}
V&=\frac{1}{4\pi\epsilon_0}\frac{q}{\dr}\\
V(\ve{r})&=\frac{1}{4\pi\epsilon_0}\int_\mathcal{V}\frac{\rho(\ve{r'})}{\dr} d\tau'
\end{align*}
Taking the gradient of the potential:
\begin{align*}
\ve{E}&=-\nabla V\\
\nabla^2V&=-\frac{\rho}{\epsilon_0}
\end{align*}

\subsubsection{Work and Energy}
%%%%%%%%%%%%%%%%%
The work needed to bring a charge $Q$ from infinity to a point $\ve{a}$ is:
\begin{align*}
W&=\int_{\infty}^{\ve{a}} \ve{F}\cdot d\ve{l}\\
&=-Q\int_{\infty}^{\ve{a}} \ve{E}\cdot d\ve{l}\\
&=QV(\ve{a})
\end{align*}
The energy in a continuous charge distribution is:
\begin{align*}
W&=\frac{1}{2}\int \rho V d\tau\\
&=\frac{\epsilon}{2} \int E^2 d\tau
\end{align*}
Where the integral is taken over all space.
\subsubsection{Conductors}
%%%%%%%%%%%%%%%%%
A perfect conductor has an unlimited supply of free charges. 
\begin{enumerate}
\item $\ve{E}=0$ and $\rho = 0$ inside a conductor
\item Any conductor is an equipotential
\item Just outside a conductor, $\ve{E}$ is perpendicular to the surface.
\end{enumerate}
If we charge up two conductors with $+Q$ and $-Q$, the potential between them is proportional to the charge $Q$ (because the electric field is proportional to $Q$), and we define the constant of proportionality capacitance:
\begin{align*}
C=\frac{Q}{V}
\end{align*}
The work done by charging a capacitor is:
\begin{align*}
W=\int_0^Q \frac{q}{C} dq &= \frac{Q^2}{2C}\\
&=\frac{1}{2}CV^2
\end{align*}
\subsubsection{Image Charges}
%%%%%%%%%%%%%%%%%
In certain special cases, a charge placed next to a grounded conductor has equivalents.
\begin{enumerate}
\item A point charge and a conducting sheet: An opposite charge in the mirror image position.
\item A point charge and a conducting sphere, or an infinite line charge and conducting cylinder: Opposite image charge and charge forms the Apollonius sphere/cylinder.
\end{enumerate}
\subsubsection{Uniqueness Theorems}
%%%%%%%%%%%%%%%%%
\begin{description}
\item[First uniqueness theorem] The solution to Laplace's equation ($\nabla^2V=0$) in some volume $\mathcal{V}$ is uniquely determined if $V$ is specified on the boundary surface $\mathcal{S}$. 
\item[Second uniqueness theorem] In a volume $\mathcal{V}$ surrounded by conductors and containing a specified charge density $\rho$, the electric field is uniquely determined if the total charge on each conductor is given. 
\end{description}
\subsection{Magnetostatics}
%%%%%%%%%%%%%%%%%
\subsubsection{Lorentz Force Law}
%%%%%%%%%%%%%%%%%
The force felt by:
\begin{enumerate}
\item A point charge $q$ moving at velocity $\ve{v}$ through a magnetic field $\ve{B}$:
\begin{align*}
\ve{F}=q\ve{v}\times\ve{B}
\end{align*}
\item A line current $\ve{I}$:
\begin{align*}
\ve{F}=I\int(d\ve{l}\times\ve{B})
\end{align*}
\item A general volume current $\ve{J}$ per unit area perpendicular to flow:
\begin{align*}
\ve{F}=\int(\ve{J}\times\ve{B})d\tau
\end{align*}
 \end{enumerate}
\subsubsection{Biot-Savart Law}
%%%%%%%%%%%%%%%%%
The magnetic field created by a steady line current:
\begin{align*}
\ve{B}(\ve{r})= \frac{\mu_0}{4\pi}I\int\frac{d\ve{l}\times\dvrhat}{\dr^2}
\end{align*}
\subsubsection{Magnetic Fields}
%%%%%%%%%%%%%%%%%
The magnetic field is divergence-free:
\begin{align*}
\ve{\nabla}\cdot\ve{B}&=0\\
\oint\ve{B}\cdot d\ve{a}&=0
\end{align*}
Taking the curl of the magnetic field gives Ampere's Law:
\begin{align*}
\ve{\nabla}\times\ve{B}&=\mu_0\ve{J}\\
\oint\ve{B}\cdot d\ve{l}&=\mu_0 I_{\text{enc}}
\end{align*}
\subsection{Electrodynamics}
%%%%%%%%%%%%%%%%%
\subsubsection{Electromotive Force}
%%%%%%%%%%%%%%%%%
For an electric field applied in a material:
\begin{align*}
\ve{J} = \frac{\ve{E}}{\rho}
\end{align*}
Where $\rho$ is the resistivity constant depending on the material.
This leads to Ohm's law:
\begin{align*}
V&=IR\\
R&=\rho\frac{l}{A}
\end{align*}
The power delivered:
\begin{align*}
P=VI=I^2R
\end{align*}
The electromotive force (emf) $\emf$ is the line integral of the force per unit charge driving the current:
\begin{align*}
\emf &=\oint \ve{E} \cdot d\ve{l}
\end{align*}
$\emf=V$ for an ideal source.
\subsubsection{Faraday's Law}
%%%%%%%%%%%%%%%%%
Faraday's law states that a changing magnetic flux $\Phi$ induces an electric field:
\begin{align*}
\emf =\oint \ve{E} \cdot d\ve{l} &= \frac{d\Phi}{dt}\\
\ve{\nabla}\times\ve{E}&=\frac{d\ve{B}}{dt}
\end{align*}
\subsubsection{Inductance}
%%%%%%%%%%%%%%%%%
If we have two current loops 1 and 2, the flux $\Phi_2$ through loop 2 is proportional to the current through loop 1:
\begin{align*}
\Phi_2 = M_{21}I_1
\end{align*}
Where $M_{21}=M_{12}$ is the mutual inductance between these two loops. We can also define an self inductance $L$, for a single loop:
\begin{align*}
\Phi &= LI\\
\emf &= -L\frac{dI}{dt}
\end{align*}
\subsection{Electric Circuits}
%%%%%%%%%%%%%%%%%

\section{Oscillations and Waves}
%%%%%%%%%%%%%%%%%
Many questions involve solving linear differential equations. For such equations, linear combinations of solutions will also be a solution.
\subsection{Oscillations}
%%%%%%%%%%%%%%%%%
\subsubsection{Simple Harmonic Motion}
%%%%%%%%%%%%%%%%%
We have a spring force, $F=-kx$.
\begin{align*}
\ddot{x}+\omega^2x&=0 \text{, where }\omega=\sqrt{\frac{k}{m}}\\
x(t)&=A \cos(\omega t+\phi)
\end{align*}
\subsubsection{Damped Oscillators}
%%%%%%%%%%%%%%%%%
In addition to the spring force, we now have a drag force $F_f=-bv$, and the total force $F=-kx-b\dot{x}$.
\begin{align*}
\ddot{x} + 2\gamma\dot{x}+\omega^2x=0
\end{align*}
Where $2\gamma=b/m$ and $\omega^2=k/m$. Let $\Omega = \sqrt{\gamma^2-\omega^2}$.
\begin{align*}
x(t)=e^{-\gamma t}(Ae^{\Omega t}+Be^{-\Omega t})
\end{align*}
\begin{description}
\item [Underdamping] $(\Omega^2<0)$
\begin{align*}
x(t)&=e^{-\gamma t}(Ae^{i\tilde{\omega}t}+Be^{-i\tilde{\omega}t})\\
&=e^{-\gamma t}C\cos(\tilde{\omega}t + \phi)
\end{align*}
Where $\tilde{\omega}=\sqrt{\omega^2-\gamma^2}$. The system will oscillate with its amplitude decreasing over time. The frequency of oscillations will be smaller than in the undamped case.
\item[Overdamping] $(\Omega^2>0)$
\begin{align*}
x(t)=Ae^{-(\gamma-\Omega)t}+Be^{-(\gamma+\Omega)t}
\end{align*}
The system will not oscillate, and the motion will go to zero for large $t$.
\item[Critical damping] $(\Omega^2=0)$

We have $\gamma=\omega$, and:
\begin{align*}
\ddot{x}+2\gamma\dot{x}+\gamma^2x=0
\end{align*}
In this special case, $x=te^{-\gamma t}$ is also a solution:
\begin{align*}
x(t)=e^{-\gamma t}(A+Bt)
\end{align*}
Systems with critical damping go to zero the quickest.
\end{description}
\subsubsection{Driven Oscillators}
%%%%%%%%%%%%%%%%%
We have to solve differential equations of this form:
\begin{align*}
\ddot{x}+2\gamma\dot{x}+ax=\sum_{n=1}^{N}{C_ne^{i\omega_nt}}
\end{align*}
We first find particular solutions for each $n$, by guessing solutions of the form $x_{p_n}(t)=Ae^{i\omega_nt}$:
\begin{align*}
-A{\omega_n}^2+2iA\gamma\omega_n+Aa=C_n \\
x_{p_n}(t)=\frac{C_n}{-{\omega_n}^2+2i\gamma\omega_n+a}e^{i\omega_nt}
\end{align*}
Using the superposition principle, the final solution is a linear combination of the general solution and the particular solutions, with the combination constants determined by initial conditions.
\subsubsection{Coupled Oscillators}
%%%%%%%%%%%%%%%%%
Normal modes are states of a system where all parts are moving with the same frequency. General strategy to find normal modes:
\begin{enumerate}
\item Write down the $n$ equations of motions corresponding to the $n$ degrees of freedom the system has.
\item Substitute $x_i=A_ie^{i\omega t}$ into the differential equations to get a system of linear equations in $A_i$, with $i=1,2,\cdots,n$
\item Non-trivial solutions exist if and only if the determinant of the matrix is zero. Solve for $\omega$, and subsequently find $A_i$
\end{enumerate}
The motion of the system can then be decomposed into linear combinations of its normal modes.
\subsubsection{Small Oscillations}
%%%%%%%%%%%%%%%%%
For an object at a local minimum of a potential well, we can expand $V(x)$ about the equilibrium point:
\begin{align*}
V(x)=&V(x_0)+V'(x_0)(x-x_0)\\
&+ \frac{1}{2!}V''(x_0)(x-x_0)^2+\cdots
\end{align*}
As $V(x_0)$ is an additive constant, and $V'(x_0)=0$ by definition of equilibrium, 
\begin{align*}
V(x) &\approx \frac{1}{2}V''(x_0)(x-x_0)^2 \\
F = -\frac{dV}{dx}&=-V''(x_0)(x-x_0) \\
%m\ddot{x}+V''(x_0)x&=V''(x_0)x_0 \\
\omega&=\sqrt{\frac{V''(x_0)}{m}}
\end{align*}
\subsection{Wave Equation}
%%%%%%%%%%%%%%%%%
A wave is a disturbance of a continuous medium that propagates with a fixed shape at constant velocity. In one dimension:
\begin{align*}
u(z,t)=u(z-vt,0)=f(z-vt)
\end{align*}
All such functions $f$ are the solutions to the wave equation:
\begin{align*}
\pd{^2 u}{x^2}=\frac{1}{v^2}\pd{^2 u}{t^2}
\end{align*}
Where $v$ is the speed of propagation.


\subsubsection{String with Fixed Ends}
%%%%%%%%%%%%%%%%%
If the equation is subject to the following initial and boundary conditions:
\begin{align*}
u_x(0, t) &= u_x(L, t) = 0 \\
u(x, 0) &= f(x) \\
u_t(x, 0) &= g(x)
\end{align*}
The solution for these conditions is:
\begin{align*}
u(x, t) &= \sum_{n=1}^{\infty}\sin{\frac{n\pi}{L} x} \; \cdot \\
&\left( a_n\sin{\frac{n \pi \alpha}{L} t}
+ b_n\cos{\frac{n \pi \alpha}{L} t} \right) \\
a_n &= \frac{2}{n \pi \alpha}\int_0^L g(x)\sin\frac{n \pi x}{L} dx \\
b_n &= \frac{2}{L}\int_0^L f(x)\sin\frac{n \pi x}{L} dx
\end{align*}

\subsubsection{D'Alembert's Solution}
%%%%%%%%%%%%%%%%%
For an infinite string, it can be proved that any solution to the wave equation can be written as a superposition of two waves of velocity $v$, one travelling to the left, the other travelling to the right. For the initial conditions:
\begin{align*}
u(x, 0) &= f(x) \\
u_t(x, 0) &= g(x)
\end{align*}
The solution of the wave equation is:
\begin{align*}
u(x, t) = \frac{1}{2} &\bigg[ f(x+vt) + f(x-vt)  \\
&+ \frac{1}{v}\int_{x-vt}^{x+vt}g(x') dx' \bigg]
\end{align*}
\subsubsection{Electromagnetic Waves}
%%%%%%%%%%%%%%%%%

\section{Optics}
%%%%%%%%%%%%%%%%%
\subsection{Geometric Optics}
%%%%%%%%%%%%%%%%%
Results from Fermat's principle of least time:
\begin{align*}
\theta_\text{incidence}&=\theta_\text{reflection} \\
n_1\sin{\theta_1}&=n_2 \sin{\theta_2}
\end{align*}
Sign convention:
\begin{itemize}
\item Light rays travel from left to right
\item $f$ is positive if surface makes rays more convergent 
\item Distances are measured from the surface (left is negative)
\item $s_o$ is negative for real objects
\item $s_i$ is positive for real images
\item $y$ above optical axis is positive
\end{itemize}
\begin{align*}
\frac{1}{f}&=\frac{1}{s_{i}}+\frac{1}{s_{o}} \\
M&=\frac{y_i}{y_o}=-\frac{s_i}{s_o}
\end{align*}
For thin lenses and mirrors:
\begin{align*}
\frac{1}{f} = \frac{2}{R}
\end{align*}
For composite thin lenses:
\begin{align*}
\frac{1}{f} = (n-1)\left(\frac{1}{R_1} + \frac{1}{R_2}\right)
\end{align*}
Lens formed by interface of two materials with different $n$:
\begin{align*}
\frac{n_2-n_1}{R} = \frac{n_2}{s_i} + \frac{n_1}{s_o}
\end{align*}
\subsection{Polarization}
%%%%%%%%%%%%%%%%%
For polarized light:
\begin{align*}
E&=E_0\cos{\theta} \\
I&=I_0\cos^2{\theta}
\end{align*}
For unpolarized light:
\begin{align*}
\langle I \rangle=I_0 \langle \cos^2{\theta} \rangle = \frac{I_0}{2}
\end{align*}
Brewster angle at which all reflected light at an interface is polarized:
\begin{align*}
\tan{\theta_i}=\frac{n_t}{n_i}
\end{align*}
\subsection{Physical Optics}
%%%%%%%%%%%%%%%%%
Interference is the superposition of wave amplitudes when waves overlap.
\subsubsection{Double Slit:}
%%%%%%%%%%%%%%%%%
Occurs when slits are of negligible width, distance between slits comparable to wavelength, such that diffraction effects are insignificant. For bright fringes:
\begin{align*}
d\sin{\theta}&=m\lambda \\
y_m&=R\frac{m\lambda}{d} \qquad m \in \mathbb{Z}
\end{align*}
For incident medium's refractive index $n_i$, reflection medium's refractive index $n_r$, if $n_i < n_r$, the reflected wave undergoes a $\frac{\pi}{2}$ phase shift.
\subsubsection{Single Slit:}
%%%%%%%%%%%%%%%%%
Occurs when size of slit is comparable to wavelength. Location of dark fringes when wavelets at distance $\frac{a}{2}$ destructively interfere:
\begin{align*}
\sin{\theta}&=\frac{m\lambda}{d} \\
y_m&=x\frac{m\lambda}{a} \qquad m \in \mathbb{Z}
\end{align*}
\subsubsection{Intensity in Diffraction Patterns}
%%%%%%%%%%%%%%%%%
For double slit interference:
\begin{align*}
I = I_{\text{max}} \cos^2\left(\frac{\pi d \sin{\theta}}{\lambda} \right)
\end{align*}
For single slit diffraction:
\begin{align*}
I = I_{\text{max}} \left [\frac{\sin(\pi a \sin \theta / \lambda)}{\pi a \sin \theta / \lambda} \right]^2
\end{align*}
Double slit including effects of diffraction:
\begin{align*}
I = I_{\text{max}} &\cos^2\left(\frac{\pi d \sin{\theta}}{\lambda} \right) \\
&\cdot  \left [\frac{\sin(\pi a \sin \theta / \lambda)}{\pi a \sin \theta / \lambda} \right]^2
\end{align*}
	
\section{Thermodynamics}
%%%%%%%%%%%%%%%%%
If two objects are in thermal equilibrium with a third system, then they are in equilibrium with each other. 

\subsection{Thermal Expansion}
For linear expansion, the change in length is:
\begin{align*}
  \Delta L = \alpha L_0 \Delta T
\end{align*}
Where $\alpha$ is the coefficient of linear expansion. For area expansion, $\beta \approx 2 \alpha $. For volume expansion, $\gamma \approx 3 \alpha$.

\subsection{Kinetic Theory of Gases}


\subsection{Thermodynamic Processes}


\subsection{Entropy}



\section{Quantum Mechanics}
%%%%%%%%%%%%%%%%%
\subsection{Schr\"{o}dinger's Equation}
%%%%%%%%%%%%%%%%%
$\Psi(x, t)$ is a complex wave function of time and position, the one-dimensional Schr\"{o}dinger's equation is given by:
\begin{align*}
i \hbar \pd{\Psi}{t} = - \frac{\hbar^2}{2m} \pd{^2 \Psi}{x^2} + V\Psi
\end{align*}
If we denote the complex conjugate of the wave function to be $\cc{\Psi}$, the conjugate of Schr\"{o}dinger's equation is:
\begin{align*}
-i \hbar \pd{\cc{\Psi}}{t} = \frac{\hbar^2}{2m} \pd{^2 \cc{\Psi}}{x^2} - V\cc{\Psi}
\end{align*}
At time $t$, the probability of finding a particle from $x=a$ to $x=b$ is:
\begin{align*}
\int_{a}^{b} |\Psi(x, t)|^2 dx = \int_{a}^{b}\Psi\cc{\Psi}dx
\end{align*}
\subsubsection{Normalization}
%%%%%%%%%%%%%%%%%
All wave functions must be normalized, so that the probability of finding the particle over all space is 1:
\begin{align*}
\int_{-\infty}^{\infty} |\Psi(x,t)|^2 dx = 1
\end{align*}
Once a function is normalized, it remains normalized as time evolves:
\begin{align*}
\frac{d}{dt} \int_{-\infty}^{\infty} \Psi \cc{\Psi} = 0
\end{align*}
\subsubsection{Expectation Values}
%%%%%%%%%%%%%%%%%
An expectation value of an observed quantity is the average of the measurement performed on many ``copies'' of the system at the same time.
\begin{align*}
\langle x\rangle &= \int_{-\infty}^{\infty} x |\Psi(x,t)|^2 dx \\
&= \int_{-\infty}^{\infty} \cc{\Psi} x \Psi dx \\
\langle p\rangle &= m\frac{d\langle x\rangle}{dt} \\
&= \int_{-\infty}^{\infty} \cc{\Psi} \left( -i\hbar \pd{}{x} \right) \Psi dx
\end{align*}
In general, the expectation value of any quantity is:
\begin{align*}
\langle Q(x, p) \rangle = \int \cc{\Psi} Q\left(x, -i\hbar\pd{}{x} \right) \Psi dx
\end{align*}

\subsection{Time Independent Solution}
%%%%%%%%%%%%%%%%%
We solve Schr\"{o}dinger's equation by separation of variables. Let:
\begin{align*}
\Psi(x, t) = \psi(x) \phi(t)
\end{align*}
Then the equation can be written as:
\begin{align*}
i\hbar \psi \pd{\phi}{t} = -\frac{\hbar^2}{2m} \pd{^2 \psi}{x^2} \phi + V \phi \psi \\
\left( \frac{i\hbar}{\phi} \pd{\phi}{t} \right) + \left( \frac{\hbar^2}{2m\psi} \pd{^2 \psi}{x^2} -V(x) \right) = 0
\end{align*}
As the two terms in the equation are independent of each other and they sum to zero, they must be constant. If we let:
\begin{align*}
E &= \frac{i\hbar}{\phi} \pd{\phi}{t} \\
\phi(t) &= e^{iE/ \hbar t}
\end{align*}
The time independent solution is given by:
\begin{align*}
-\frac{\hbar^2}{2m}\pd{^2\psi}{x^2} + V(x) \psi = E \psi\
end{align*}
If we define the Hamiltonian operator $\operator{H} = -\frac{\hbar^2}{2m}\pd{^2}{x^2} + V$,
\begin{align*}
\operator{H} \psi = E\psi
\end{align*}
\end{multicols*}
\end{document}
